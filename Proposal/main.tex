\documentclass[a4paper]{article}

%% Language and font encodings
\usepackage[english]{babel}
\usepackage[utf8x]{inputenc}
\usepackage[T1]{fontenc}

%% Sets page size and margins
\usepackage[a4paper,top=3cm,bottom=2cm,left=3cm,right=3cm,marginparwidth=1.75cm]{geometry}

%% Useful packages
\usepackage{amsmath}
\usepackage{graphicx}
\usepackage[colorinlistoftodos]{todonotes}
\usepackage[colorlinks=true, allcolors=blue]{hyperref}

\title{Project Proposal \\ EE 381K: Convex Optimization }

\author{Soovadeep Bakshi (soovadeep.bakshi@utexas.edu), Rohit Arora (arorarohit@utexas.edu)}

\begin{document}
\maketitle

% \begin{abstract}
% Your abstract.
% \end{abstract}

\section{Overview}

In this project, we study the generalized lasso problem, and more specifically, the fused lasso problem and implement efficient algorithms for achieving the optimal solution.

\section{Introduction}

The generalized lasso problem is defined as \cite{Tibs16}:
\[\min_{\beta} f(\beta) + \lambda ||D \beta ||_1 \]
where $ f: \mathbf{R}^n \rightarrow \mathbf{R} $ is convex and smooth, and $D \in \mathbf{R}^{mxn}$ is a matrix which determines the penalty function. If $D = I$, the problem can be interpreted as the usual lasso implementation. With different structures of the matrix D, the problem can be modified for different types of trend fitting.

In this project, we are going to specifically look at the fused lasso variant introduced by Tibshirani \textit{et.\ al.} \cite{Tibs05}. The fused lasso problem is a special case of the generalized lasso, defined as: 
\[||D\beta||_1 = \sum ^{n-1} _{i = 1} |\beta_i - \beta_{i+1}|\]
for the 1D case. It is evident that this formulation generates sparsity not in the solution (like the usual lasso), but in the adjacent differences, i.e., it encourages adjacent members of the solution vector to be equal to each other. The function $f(\beta)$ can be a Gaussian or a Logistic loss function as both are smooth, convex functions in $\beta$. The fused lasso can be used for total variation denoising in 1D.

Additionally, the fused lasso can be defined for total variation denoising over a graph, $G = ({1,\cdots n}, E)$, and is defined as:
\[||D\beta||_1 = \sum _{(i,j) \in E} |\beta_i - \beta_{j}|\]
which encourages the solution to have equal components across the any $(i,j) \in E$. This can be especially useful for denoising images as images can be represented as a 2D grid graph.

\section{Methods}

We seek to implement gradient methods for solving the fused lasso, namely the primal subgradient method, the dual proximal gradient method, as well as the dual path algorithm presented by Tibshirani and Taylor \cite{Tibs11}. Arnold and Tibshirani \cite{Tibs16} have shown a fast implementation of the dual path algorithm, which we seek to reproduce.

\bibliographystyle{alpha}
\bibliography{sample}

\end{document}